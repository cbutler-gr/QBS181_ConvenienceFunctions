\documentclass[a4paper]{book}
\usepackage[times,inconsolata,hyper]{Rd}
\usepackage{makeidx}
\usepackage[utf8]{inputenc} % @SET ENCODING@
% \usepackage{graphicx} % @USE GRAPHICX@
\makeindex{}
\begin{document}
\chapter*{}
\begin{center}
{\textbf{\huge Package `ConvenienceFunctions'}}
\par\bigskip{\large \today}
\end{center}
\inputencoding{utf8}
\ifthenelse{\boolean{Rd@use@hyper}}{\hypersetup{pdftitle = {ConvenienceFunctions: Convenience functions for R for QBS181}}}{}\ifthenelse{\boolean{Rd@use@hyper}}{\hypersetup{pdfauthor = {Carly Bobak}}}{}\begin{description}
\raggedright{}
\item[Type]\AsIs{Package}
\item[Title]\AsIs{Convenience functions for R for QBS181}
\item[Version]\AsIs{0.1.0}
\item[Author]\AsIs{Carly Bobak}
\item[Description]\AsIs{We proide general utilities for common taks in data wrangling}
\item[License]\AsIs{MIT}
\item[Depends]\AsIs{R (>= 3.5.0)}
\item[Encoding]\AsIs{UTF-8}
\item[LazyData]\AsIs{true}
\item[Imports]\AsIs{stats,
ggplot2}
\item[RoxygenNote]\AsIs{7.1.2}
\end{description}
\Rdcontents{\R{} topics documented:}
\inputencoding{utf8}
\HeaderA{completeFun}{Drop NAs by Columns}{completeFun}
%
\begin{Description}\relax
Remove NAs based on specified columns in the data
\end{Description}
%
\begin{Usage}
\begin{verbatim}
completeFun(data, desiredCols)
\end{verbatim}
\end{Usage}
%
\begin{Arguments}
\begin{ldescription}
\item[\code{data}] data.frame object of variations

\item[\code{desiredCols}] list of columns from which incomplete cases should be dropped
\end{ldescription}
\end{Arguments}
%
\begin{Value}
dataframe with removed observations
\end{Value}
%
\begin{Examples}
\begin{ExampleCode}
data<-data.frame(a=1:4,b=c("a","b","c","d"),c=c(NA,"keep",NA,"keep"))
completeFun(data,c("c"))

\end{ExampleCode}
\end{Examples}
\inputencoding{utf8}
\HeaderA{factorial}{Factorial}{factorial}
%
\begin{Description}\relax
Function to calculate the factorial of a variable
\end{Description}
%
\begin{Usage}
\begin{verbatim}
factorial(x)
\end{verbatim}
\end{Usage}
%
\begin{Arguments}
\begin{ldescription}
\item[\code{x}] numeric vector
\end{ldescription}
\end{Arguments}
%
\begin{Value}
numeric value of factorial
\end{Value}
%
\begin{Examples}
\begin{ExampleCode}
factorial(5)

\end{ExampleCode}
\end{Examples}
\inputencoding{utf8}
\HeaderA{gm\_mean}{Geometric mean}{gm.Rul.mean}
%
\begin{Description}\relax
Function to calculate the geometric mean of a variable
\end{Description}
%
\begin{Usage}
\begin{verbatim}
gm_mean(x, na.rm = TRUE)
\end{verbatim}
\end{Usage}
%
\begin{Arguments}
\begin{ldescription}
\item[\code{x}] numeric vector
\end{ldescription}
\end{Arguments}
%
\begin{Value}
numeric value of geometric mean
\end{Value}
%
\begin{Examples}
\begin{ExampleCode}
x<-c(1,1,3,5,6,6)
gm_mean(x)

\end{ExampleCode}
\end{Examples}
\inputencoding{utf8}
\HeaderA{Modes}{Mode}{Modes}
%
\begin{Description}\relax
Function to calculate the mode of a variable
\end{Description}
%
\begin{Usage}
\begin{verbatim}
Modes(x)
\end{verbatim}
\end{Usage}
%
\begin{Arguments}
\begin{ldescription}
\item[\code{x}] numeric vector
\end{ldescription}
\end{Arguments}
%
\begin{Value}
numeric vector of modes
\end{Value}
%
\begin{Examples}
\begin{ExampleCode}
x<-c(1,1,3,5,6,6)
Modes(x)

\end{ExampleCode}
\end{Examples}
\inputencoding{utf8}
\HeaderA{nonUnique}{Non-unique}{nonUnique}
%
\begin{Description}\relax
Function that returns all non-unique values in a vector
\end{Description}
%
\begin{Usage}
\begin{verbatim}
nonUnique(x)
\end{verbatim}
\end{Usage}
%
\begin{Arguments}
\begin{ldescription}
\item[\code{x}] numeric or character vector
\end{ldescription}
\end{Arguments}
%
\begin{Value}
numeric or character vector of non-unique values
\end{Value}
%
\begin{Examples}
\begin{ExampleCode}
x<-c(1,1,3,5,6,6)
nonUnique(x)

\end{ExampleCode}
\end{Examples}
\printindex{}
\end{document}
